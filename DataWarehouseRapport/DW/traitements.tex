\section{Traitements}

\paragraph{} Les données que nous aurons besoin de stocker dans l’entrepôt concernent : 

\begin{itemize}
    \item Les plateaux commandés : Ceux-ci sont stockés dans la table repas de notre base de données.
    \item Clients ayant passé des commandes de plateaux : Tuples de la table Client.
    \item Date : Il s’agit de la dimension temps (Jour, mois, années) permettant de poser un suivi des activités du business à travers le temps.
\end{itemize}

\paragraph{} Afin de permettre des analyses plus détaillées, nous choisissons d’étendre le modèle du premier projet en ajoutant les notions de Fournisseur et du Local. Pour souligner l’hétérogénéité des données des entrepôts nous supposons que ces ajouts proviennent de sources de données différentes ;
\newline
\textbf{Attention : les éléments suivants sont déprécies depuis les derniers changements dans la conception.}
\begin{itemize}
    \item Livreur : Informations relatives aux livreurs de produits du terroir.
    \item Locaux : Sortes de magasins servant à recueillir, stocker, et envoyer les commandes aux domiciles des clients.

\end{itemize}

\subsection{Actions / Opérations}
Pour parvenir à des stratégies cohérentes et prometteuses d’évolution, nous proposons un petit ensemble de démarches décisionnelles stratégiques (opérations). Pour chacune de ces démarches, une entité dirigeante fictive propose un certain nombre d’idées d’améliorations.

\paragraph{}Il s’agira ensuite de déduire les faits et les traitements (analyses) à partir de ces propositions.
\begin{itemize}
    \item a) Opération de communication et marketing: Ce volet vise à promouvoir les services de l’entreprise, l’intérêt étant de cibler une clientèle de plus en plus grande, ainsi que de fidéliser les clients déjà existants de l’entreprise.
        \\
        \begin{itemize}
            \item 1-Communication : Les dirigeants de l’entreprise souhaiteraient organiser des foires de dégustation des différents produits du terroir, et en profiter pour parler de leur entreprise.  Pour une meilleure efficacité de cette opération de communication, ils se demandent quelles sont les régions prioritaires où pourront avoir lieux ces foires (Les régions les moins intéressées par la restauration du terroir), et quels sont les produits les moins connus des consommateurs par région.
            \item 2-Marketing : Afin d’améliorer les ventes et les chiffres d’affaire de l’entreprise, les idées suivantes sont proposées : 
                \\
                \begin{itemize}
                    \item Création de menus (plateaux) formés des assortiments et associations des produits les plus populaires (couple, triplet de produits) auprès des clients par région et par saison.
                    \item Proposer des offres d’abonnement/réductions aux clients les plus fidèles.
                    \item Création de menus à moindre prix pour une certaine tranche d’âge de clients.
                    \item Création de menus (plateaux) formés des assortiments et associations des produits les plus populaires (couple, triplet de produits) auprès des clients par région et par saison.
                    \item Proposer des offres d’abonnement/réductions aux clients les plus fidèles.
                    \item Création de menus à moindre prix pour une certaine tranche d’âge de clients.
                \end{itemize}
        \end{itemize}
    \\
    \item b) Opération ouverture de nouveaux locaux en optimisant les trajets de livraison : 
Une équipe de recherche développe des solutions d’optimisation des trajets du service de livraison. En complément aux heuristiques développées, ils proposent afin de réduire le coût engendré par la livraison des produits, les dispositions suivantes : 
Pour chaque région, établir les nouveaux locaux à des emplacements proches des fournisseurs et du périmètre le plus desservi (clients les plus souvent livrés)
    \\
    \item c) Opération optimisation stockage des produits avant leurs livraisons : 
L’équipe des stocks suggère également d’identifier les produits les plus demandés par région afin de gérer les approvisionnements (décider des quantités à stocker), et ainsi éviter l’épuisement de ces produits.
\end{itemize}

\newpage 
\subsection{Requêtes analytiques} 
\textbf{Operation optimisation gain :}
\begin{itemize}
    \item Quelles sont les villes ayant le moins d’habitants qui ont déjà commandé un plateau repas ? (Ou les villes dont les habitants n’ont jamais commandé de plateaux repas)
    \item Quels sont les produits les moins commandés par régions ? 
    \item Quels sont les produits les plus souvent associés dans une commande ?
    \item Quels sont les produits commandés par les clients ayant un âge compris dans une certaine tranche ?
    \item Opération nouveaux locaux/optimisation trajets et route 
Quels sont les clients effectuant des commandes régulièrement (Tous les x jours) ?
    \item Quels sont les fournisseurs dont les produits sont les plus appréciés par région ?
\end{itemize}

\textbf{Opération optimisation stockage :}
\begin{itemize}
    \item Quels sont les produits disponibles dans un local donné ?
    \item Quel jour un local de stockage donné a été saturé ? \item Quelle est la capacité restante (aujourd’hui) d’un local donné ?
\end{itemize}

\paragraph {Ordre d'importance des opérations} 

\begin{itemize}
    \item a)-Commandes/Ventes de plateau
    \item c)-Approvisionnement et Stockage
    \item b)-Ouverture de nouveaux locaux de stockage
\end{itemize}



