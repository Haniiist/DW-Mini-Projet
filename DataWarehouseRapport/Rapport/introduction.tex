\section{Introduction}

\paragraph{} Le sujet auquel nous nous sommes penchés lors du mini-projet objet-relationnel proposait d’éventuellement asseoir une application de commandes de plateaux repas constitués de produits du terroir. 

\paragraph{} Nous adoptons dans le présent document une vision d’entreprise. Les activités d’un organisme dans son terrain d’expertise se déclinent en un corpus important de données regroupant l’historique de sa production et de ses évènements passés (faits, ventes, clients, produits…).
L’évolution du volume de ces données est parallèle à un savoir-faire également grandissant au fil de son expérience. La rapidité de cette évolution dépend de la capacité des organismes à tirer profit de leurs expériences et de la qualité des décisions prises par sa direction.

\paragraph{} Il s’agit maintenant de supposer la prospérité des services de livraison du produit du terroir, l’analyse suivante consiste à proposer les potentielles directions stratégiques que pourraient emprunter les entreprises proposant le service en question afin d’améliorer leurs productivités, ainsi que les analyses à effectuer pour mener à bien cette démarche.